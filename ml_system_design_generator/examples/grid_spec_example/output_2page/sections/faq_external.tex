
\section{External FAQ --- Operations \& End-User Questions}\label{faq_external}

\begin{table}[htbp]
\centering
\caption{External FAQ --- Operations \& End-User Questions}\label{tab:faq-external}
\begin{tabularx}{\textwidth}{c >{\raggedright\arraybackslash}p{0.28\textwidth} >{\raggedright\arraybackslash}X}
\toprule
\textbf{\#} & \textbf{Question} & \textbf{Answer} \\
\midrule
1 & What does it show during a hurricane or wildfire? & Ranked feeder-level risk heat maps with affected-customer counts and calibrated uncertainty bands. \\
\addlinespace
2 & How far ahead and how often? & Day-ahead forecasts refresh every 15~minutes; uncertainty bands widen with lead time. \\
\addlinespace
3 & Can I trust it in unprecedented storms? & Ensemble methods provide calibrated uncertainty, but tail-event limitations exist. Operators retain authority. \\
\addlinespace
4 & How does it integrate with OMS/DMS? & FastAPI dashboards embed alongside existing consoles; OMS/DMS widget roadmap planned. \\
\addlinespace
5 & Does it cover wildfire PSPS decisions? & Yes---fuel moisture, wind, and asset-condition scores feed the same feeder-level risk framework for PSPS go/no-go governance. \\
\addlinespace
6 & What data does it need? & OMS outage history, SCADA telemetry, GIS/asset registry, and weather feeds. New territories require a data-quality assessment. \\
\bottomrule
\end{tabularx}
\end{table}


\paragraph{Supplementary Materials.} For engineering, compliance, and business details, see Appendix: Internal FAQ.